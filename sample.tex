\documentclass{sokuten}

\title{記事タイトル}
\author{著者名}
\date{}

\begin{document}
\maketitle

\section{執筆時のお約束}
以下のお約束を守って書いてください。
\begin{itemize}
	\item 句読点は「,」「。」を使ってください
	\item 引用符は基本的に「」を推奨します。
	\begin{itemize}
		\item クォーテーションを使う場合は`` ''を使ってください
	\end{itemize}
	\item 図を挿入するときは必ずグレースケールに変換してください
	\begin{itemize}
		\item 対応するフォーマットはPDF, PNGです
	\end{itemize}
	\item ソースコードを挿入するときは\verb|listing|環境を使ってください\footnote{\url{https://ctan.org/pkg/listings}}
	\item 図表を入れたときは文中で参照するようにしましょう
	\begin{itemize}
		\item 図\ref{img:sample}はおるみん党ステッカーのデザインです
	\end{itemize}
\end{itemize}
\section{サンプル}
松尾芭蕉はその著書「奥の細道」の冒頭で次のように書き記した

\begin{quote}
月日は百代の過客にして、行き交う人もまた旅人なり
\end{quote}

これは松尾芭蕉が我らが党首おるみんのように時空間を把握することができたことを示している。

\begin{lstlisting}
int main(int argc, char *argv[]) {
	printf("Hello World!\n");
	return 0;
}
\end{lstlisting}

\begin{figure}[tb]
	\centering
	\includegraphics[width=0.8\linewidth]{images/oruminparty_sticker.png}
	\caption{おるみん党ステッカー}
	\label{img:sample}
\end{figure}

\end{document}
