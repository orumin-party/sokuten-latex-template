\documentclass[b5j,papersize,twocolumn]{ltjsarticle}

\usepackage{luatex85,luatexja}
\usepackage{graphicx, xcolor}
\usepackage{mathtools, amssymb}
% \usepackage[utf8x]{luainputenc}
\usepackage{textcomp}
\usepackage[unicode]{hyperref}
\usepackage{titlesec, titletoc}
\usepackage{fancyvrb}
\usepackage{fvextra}
\usepackage{listings}

% 欧文組版のマイクロタイポグラフィー(細かな読み易さの調整)を有効化
\usepackage{microtype}

% フォントのエンコード
\usepackage[T1]{fontenc}

% OTF フォント(和文フォント)
\usepackage{luacode}
\usepackage{luatexja-ruby}
\usepackage{luatexja-otf}
\usepackage[noembed,deluxe,jis2004]{luatexja-preset} % 源ノ角/源ノ明朝フォント

% フォント設定(数式)
\usepackage[math-style=ISO,bold-style=ISO]{unicode-math}

% フォント設定(欧文)
% main ... 本文フォント。ローマン体。\textrm で使用
% sans ... ヒゲのないフォント。サンセリフ体(和文で言うゴシック体)。\textsf で使用
% mono ... 固定幅フォント。タイプライタ体。ソースコードのリストに使う。\texttt で使用
% math ... 数式フォント
%
\setmathfont{TeX Gyre Termes Math}
%\setmathfont{TeX Gyre Pagella Math}
\setmainfont[Ligatures=TeX, Scale=0.95]{TeX Gyre Termes}
%\setmainfont[Ligatures=TeX, Scale=0.95]{TeX Gyre Pagella}
\setsansfont[Ligatures=TeX, Scale=0.95]{TeX Gyre Heros}
%\setsansfont[Ligatures=TeX, Scale=0.9]{TeX Gyre Adventor}
\setmonofont[Ligatures=TeX, Scale=1]{TeX Gyre Cursor}

% TeX Gyre フォントと OTF フォントの共存
\renewcommand{\bfdefault}{bx}
\renewcommand{\headfont}{\gtfamily\sffamily\bfseries}

\usepackage{orumin}

\title{記事タイトル}
\author{著者名}
\date{}

\begin{document}
\pagestyle{empty}
\maketitle

\section{執筆時のお約束}
以下のお約束を守って書いてください。
\begin{itemize}
	\item 句読点は「,」「。」を使ってください
	\item 引用符は基本的に「」を推奨します。
	\begin{itemize}
		\item クォーテーションを使う場合は`` ''を使ってください
	\end{itemize}
	\item 図を挿入するときは必ずグレースケールに変換してください
	\begin{itemize}
		\item 対応するフォーマットはPDF,PNGです
	\end{itemize}
	\item ソースコードを挿入するときは\verb|listing|環境を使ってください\footnote{\url{https://ctan.org/pkg/listings}}
	\item 図表を入れたときは文中で参照するようにしましょう
	\begin{itemize}
		\item 図\ref{img:sample}はおるみん党ステッカーのデザインです
	\end{itemize}
\end{itemize}
\section{サンプル}
松尾芭蕉はその著書「奥の細道」の冒頭で次のように書き記した

\begin{quote}
月日は百代の過客にして、行き交う人もまた旅人なり
\end{quote}

これは松尾芭蕉が我らが党首おるみんのように時空間を把握することができたことを示している。
\begin{figure}[tb]
	\centering
	\includegraphics[width=0.8\linewidth]{images/oruminparty_sticker.png}
	\caption{おるみん党ステッカー}
	\label{img:sample}
\end{figure}

\begin{lstlisting}
int main(int argc, char *argv[]) {
	printf("Hello World!\n");
	return 0;
}
\end{lstlisting}
\end{document}
